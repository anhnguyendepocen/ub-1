\label{sec:results}
The different estimates of the ETI in table \ref{tab:elasticities} show a huge discrepancy from the seminal studies by \citet{feldstein1995effect} over other U.S. studies \citep{gruber2002elasticity} to the later studies for Denmark \citep{chetty2011adjustment,kleven2014estimating,kreiner2016tax}.
\begin{table}[h]
  \centering
  \footnotesize
    \begin{tabular}{lcccrcc}
\toprule
{}          & $\hat{\varepsilon}$ & Income group    & Method              & N         &  Period   & Country \\
\midrule
Feldstein (1995)          & 1.04  & $\sim\$100,000$ & OLS DD              & 3,792     & 1985-1988 & U.S.    \\
Gruber \& Saez (2002)     & 0.40  & $>\$10,000$     & IV Panel Regression &$\sim$60,000&1979-1990 & U.S.    \\
\ditto                    & 0.57  & $>\$100,000$    & \ditto              & \ditto    & \ditto    & \ditto  \\
Kleven \& Schultz (2014)  & 0.05  & wage earners    & IV Panel Regression & 29,668,870& 1984-2005 & DK      \\
\ditto                    & 0.09  & self-employed   & \ditto              & 1,646,270 & \ditto    & \ditto   \\
\ditto                    & 0.11  & all taxpayers   & \ditto              & 11,799,628& 1984-1990 & \ditto  \\
\ditto                    &0.2-0.3& \ditto          & IV DD         & $\sim$3,000,000 & 1986-1989 & \ditto  \\
Kreiner et al (2016)      & 0.08  & highest quartile& OLS DD              & 328,679   & 2009-2010 & DK      \\
\ditto                    & 0.00  & \ditto          & \ditto              & \ditto    & 2009-2010*& \ditto  \\
Chetty et al (2011)       & 0.00  & wage earners    & IV Panel Regression & 8,302,905 & 1994-2001 & DK      \\
\bottomrule
\end{tabular}
% \begin{table}[h]
%   \centering
%   \footnotesize
%     \input{04_tables/elasticities}
%   \caption{Example of table}
%   \label{tab:elasticities}
% \end{table}

  \caption{Estimated elasticity of taxable income in different studies. *excl. N09, D09 \& J10.}
  \label{tab:elasticities}
\end{table}
\\
An estimate of $\hat{\varepsilon}>1$ would refer to U.S. being on the wrong side of the laffer curve prior to 1986 such that reducing the tax rate (for the income group in question) would actually raise collected tax revenue. Besides from working with a very small sample-size \citet{feldstein1995effect} does not control for issues with mean reversion and other income distribution changes nor solve endogeneity problems. Controlling for theese issues by using IV panel regression and including controls for baseline income \citet{gruber2002elasticity} estimate a significantly smaller ETI covering more years around the TRA86 though the result is very fragile to different specifications. Nonetheless, most other reliable estimates also find the ETI to be in the range 0.12-0.40 for the U.S. \citep{saez2012elasticity}.

Throughout the surveyed studies there is a clear indication that the responses to changes in the MTR are more evident for high-income earners as they both might have better knowledge as well as better possibilities to react such as deductibles \citep{saez2012elasticity}. Thus, the overall effect will be lower if estimated for all taxpayers. Furthermore, if looking at responses to a broader tax base in terms of all types of incomes the elasticity of the rate of the personal income tax is down from 0.40 to 0.12 \citep{gruber2002elasticity}. While partly mechanical due to the tax base being wider it also shows the presence of fiscal externalities due to avoidance. More so, the discrepancy in the estimate between taxable income and broad income is found to be much smaller in Denmark \citep{kleven2014estimating}. Besides from the fact that labour income to a greater degreee is the main income component in Denmark, this suggests that options for avoidance or evasion is much smaller than in the U.S. which could help explain the smaller estimates of EIT on Danish data.

Using a similar approach for Denmark but with rich individul background data available, \citet{kleven2014estimating} takes advantag of the nonlinear variations in the various tax reforms in the 80s and 90s as well of the fact that inequality is close to constant over the period. They find quite modest responses to the MTR over the period for wage earners as well as self-employed. The estimate is a little higher for the sub-sample around the thorough 1987 tax reform while at it's highest for the one DD estimation from 1986-1989 that seems to capture most of the effect. This corresponds with the concept that larger changes in incentives are more likely to cause a reaction that reveals the long-run elasticities by overcoming frictions such as searching and other switching costs but likewise a simple attention cost \citep{chetty2012bounds}.

Nonetheless, a sizable decrease in the MTR from 63 pct. to 56 pct. for the highest-earning quartile of the full-time employees was only found to have a minor effect \citep{kreiner2016tax}, though estimation on subgroups shows that within the treated group the ETI estimate grows continously from 0 for the lowest incomes to 0.25 for the top one percent incomes. Even more so, all of the effects could be assigned to intertemporal income shifting, leaving no real responses when leaving out the few months around the implementation. Reversely, the estimate is 0.80 when only taking december 2009 and january 2010 into account.

Using bunching methods \citet{chetty2011adjustment} find substantial elasticities in the proximity of the kinks in the MTR as well as for self-employed and secondary earners. Nonetheless, the overall elasticity is insignificant for the smaller tax reforms analyzed.

A lower EIT estimate in Denmark can partly be due to larger frictions on an institutional level such as a labour market dominated by collective agreements. This means adjustments take longer as on the collective level as agreements are only being renegotiated every 2-3 years and well as on the individual level where career changes might be necessary, thus, if effects are spread out over more than a the three years included in the specifications, estimates will be downward biased \citep{chetty2011adjustment}.
