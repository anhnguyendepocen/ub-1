%%% File encoding is ISO-8859-1 (also known as Latin-1)
%%% You can use special characters just like ä,ü and ñ

% Input encoding is 'latin1' (Latin 1 - also known as ISO-8859-1)
% CTAN: http://www.ctan.org/pkg/inputenc
%
% A newer package is available - you may look into:
% \usepackage[x-iso-8859-1]{inputenc}
% CTAN: http://www.ctan.org/pkg/inputenx
\usepackage[utf8]{inputenc}
\usepackage{import}
\usepackage{dsfont}
% Font Encoding is 'T1' -- important for special characters such as Umlaute ü or ä and special characters like ñ (enje)
% CTAN: http://www.ctan.org/pkg/fontenc
\usepackage[T1]{fontenc}

% Language support for 'english' (alternative 'ngerman' or 'french' for example)
% CTAN: http://www.ctan.org/pkg/babel
\usepackage[british,UKenglish,USenglish,english,american]{babel}
\usepackage{csquotes}

% Extended graphics support
% There is also a package named 'graphics' - watch out!
% CTAN: http://www.ctan.org/pkg/graphicx
\usepackage{graphicx}

%customized line spacing
\usepackage{setspace}
\usepackage{dirtytalk}
%Inclusion of pdf's
\usepackage{pdfpages}


%Create random text
\usepackage{blindtext} % \Blindtext \blindtext \blindtext[10]
\usepackage{lipsum} % \lipsum[10]

%Create publication quality tables
\usepackage{booktabs}

%biblatex is used for creating bibliography
\usepackage[style=authoryear, backend=bibtex8, natbib=true, maxcitenames=2]{biblatex}

\usepackage{pdfpages}

%More comprehensive math typing
\usepackage{amssymb}


\usepackage{mathtools}
%To allow new math operators
\usepackage{amsmath}

%Nice matrix features
\usepackage{physics}
\usepackage[ruled,vlined]{algorithm2e}
%Abbreviations
\usepackage[hyperref=true]{acro}

\usepackage{enumerate}

\usepackage{xcolor}
\usepackage{multirow}
\usepackage{tabularx}
\usepackage[figuresright]{rotating}

\usepackage{subcaption}
\usepackage{caption}
\usepackage{listings}
\usepackage{kbordermatrix}% http://www.hss.caltech.edu/~kcb/TeX/kbordermatrix.sty
\usepackage{bm}
\usepackage{threeparttable}

\usepackage[osf, sc]{mathpazo} % Use the Palatino font
\usepackage{placeins}
\usepackage{marginnote}

\usepackage[super]{nth} % Write 1st, 2nd as \nth{1}, \nth{2} etc.

\usepackage{multicol}
\usepackage{epigraph}
