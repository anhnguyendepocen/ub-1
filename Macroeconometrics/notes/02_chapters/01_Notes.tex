\epigraph{""}
{\textit{}}

\section{Theoretical issues}
\begin{multicols}{2}
 \noindent
 \textbf{Convergence in distribution}
 \begin{align*}
  \hat{\beta}_{OLS}         & \xrightarrow{d} \textrm{N}(\beta,V(\beta)) \\
  (\hat{\beta}_{OLS}-\beta) & \xrightarrow{d} \textrm{N}(0,V(\beta))
 \end{align*}
 \begin{align*}
  \hat{\beta}_{OLS} & = (x'x)^{-1}x'y         \\
                    & = \beta + (x'x)^{-1}x'u
 \end{align*}\noindent
 \textbf{Asymptotic independence}\par
 If independent, the joint probability equals the product of probabilities, i.e.
 \begin{align*}
  P(A\cap B) = P(A)*P(B)
 \end{align*}
\end{multicols}


\section{ARIMA modelling}
\begin{multicols}{2}\noindent
 \textbf{Univariate TS strategy}\par
 Not based on Economic Theory\\
 (reduced form modelling $\rightarrow$ the past is informative about the present)
 \begin{align*}
  y_t = f(y_{t-1},y_{t-2},y_{t-3}\cdot)
 \end{align*}
 The performance in terms of \textbf{forecasting} is key!\par
 If you want to learn about main drivers, you need to use other models.\\
 \\
 \textbf{1. Stationary time series:}\par
 Sometimes \nth{1} difference is required to ensure stationarity.\\
 \\
 \textbf{2. Identification:}\par
 Specify the model that explains the behaviour the best
 \begin{itemize}
  \item[AR:] Autoregressive model
  \item[MA:] Moving Average model
  \item[ARMA:] AutoRegressive Moving Average model
 \end{itemize}
 To initially decide between them, plot the
 \begin{itemize}
  \item \textbf{ACF:} The Simple Autocorrelation Function.\\
        Check if there is persistence/memory:
        \begin{align*}
         \frac{Cov(x_t,x_{t+k})}{\sqrt{V(x_t)}\sqrt{V(x_{t+k})}},\textrm{     for }k=0,\pm1,\pm2,\cdots
        \end{align*}
        Related to the asymptotic confidence interval
        \begin{align*}
         \left(\frac{-1.96}{\sqrt{(T)}}, \frac{1.96}{\sqrt{(T)}}  \right)
        \end{align*}
        or degrees of freedom corrected
        \begin{align*}
         \left(\frac{-1.96}{\sqrt{(T-k)}}, \frac{1.96}{\sqrt{(T-k)}}  \right)
        \end{align*}
  \item \textbf{PACF:} The Partial Autocorrelation Function\\
        e.g. The The \nth{3} partial autocorrelation coeffcient of the PACF, $\phi_{33}$:
        \begin{align*}
         \tilde{x}_t=\phi{31}\tilde{x}_{t-1}+\phi{32}\tilde{x}_{t-2}+\phi{33}\tilde{x}_{t-3}+\varepsilon_t
        \end{align*}
  \item [$\rightarrow$] Using both functions helps us to make an initial specification.
        \begin{itemize}
         \item Helps us to decide on the number of lags.
         \item Too many lags can cause multicolliniarity.
        \end{itemize}
 \end{itemize}
 Better to choose an $f(t)\neq0$ to test it.\\
 \\
 \textbf{3. Estimation}\par
 ..
 \\
 \textbf{4. Validation}\par
 How well is the model specified?
 \begin{itemize}
  \item
  \item Residuals
 \end{itemize}
 Re-specify if not correctly validated (repeat step 2-4)\\
 \\
 \textbf{5. Forecasting}\par
 When model is valid, use it to forecast.


 \subsection{AR(p) process}\noindent
 Stationarity conditions for the AR model:
 \begin{itemize}
  \item[1.] The deterministic component $f$ does not depend on $t$ (rules out MA)
  \item[2.] All roots of the autoregressive polynomial must, in module, be greater than one, e.g.\\
 \end{itemize}
 \begin{align*}
  x_t                                                     & =\phi_1x_{t-1}+\varepsilon_t                                                  \\
  \underbrace{(1-\phi_1 (L))}_\text{AR polynomial}x_{t-1} & =\varepsilon_t,\ \ \ L>1\textrm{ required!}                                   \\
  V(x_t)                                                  & =V(\phi_1x_{t-1})+V(\varepsilon_t)                                            \\
  \sigma^2_x                                              & =\phi_1^2 \sigma_x^2 +                                                        \\
  \sigma_x^2                                              & =\frac{1}{1-\sigma_1^2}\sigma^2_\varepsilon\textrm{,   requires }\sigma^2_1<1
 \end{align*}


 \subsection{MA(q) model}\noindent
 If MA is invertible
 \begin{align*}
  \Rightarrow MA(q) & \equiv AR(\infty)                       \\
                    & \simeq AR(p),\ \ \ p\ \textrm{is large}
 \end{align*}
 Use the $\hat{\varepsilon}_t^0$ estimation algorithm.

\end{multicols}



\section{Order of Integration Analysis}
\begin{multicols}{2}\noindent
 \textbf{\#1. Iterative Strategy:} DIY.\\
 \\
 \textbf{\#2. Model Selection Strategy:}\\
 Automated procedure using AIC or BIC.
 \begin{itemize}
  \item $p_max=3$
  \item $q_max=3$
 \end{itemize}
 "Autometrics".\\
 \\
 Models that rely on $x_t$ being stationary and invertible (for consistent $\hat{\beta}$):
 \begin{align*}
  \begin{array}{lc|c}
             & ACF                 & PACF                \\
   AR(p)     &                     & x_{(p)}             \\
   MA(q)     & x_q                 &                     \\
   ARMA(p,q) & ?_{\hat{p},\hat{q}} & ?_{\hat{p},\hat{q}}
  \end{array}
 \end{align*}
 Usually $x_t$ is non-stationary.
 \begin{itemize}
  \item[$\rightarrow$] $\Delta^d x_t, d=0,1,2,3,\cdots$ is stationary?
        \begin{itemize}
         \item[$\rightarrow$] $\hat{\beta}$ is consistent!
        \end{itemize}
  \item Integrated of order "d".
        \begin{itemize}
         \item $prices_t$
         \item $\Delta prices_t \simeq$ Inflation.
         \item $\Delta^2 prices_t \simeq$ Acceleration prices.
        \end{itemize}
 \end{itemize}
 A non-stationary proces $\simeq$ Random walk process features (uniroot).
 \begin{align*}
  x_t & = \phi x_{t-1}+\varepsilon_t,\ \ \ \ \ \phi=1    \\
      & = x_0 + \sum\limits_{j=0}^{t-1}\varepsilon_{t-j}
 \end{align*}
 With the variance
 \begin{align*}
  V(x_t) & = V(x_0) + V\left(\sum\limits_{j=0}^{t-1}\varepsilon_{t-j}\right) \\
         & \vdots                                                            \\
         & = \sigma^2_{\varepsilon}t
 \end{align*}

 \subsection{Unit root tests}
 \textbf{t-Student distrubution}
 \begin{itemize}
  \item Zero mean.
  \item Broad tails with a low degree of freedoms $k$.
  \item[$\rightarrow$] Tends towards the normal distribution when $k\rightarrow\infty$.
 \end{itemize}
 \textbf{Dickey-Fuller (DF) test}
 \begin{itemize}
  \item Only valid for an AR(1) process!
  \item Statistical inference: Objective:
        \begin{itemize}
         \item Test statistics:\\
               $H_0:\ x_t \sim I(1)$\\
               $H_1:\ x_t \sim I(0)$
         \item "Confirmatory analysis":\\
               $H_0:\ x_t \sim I(0)$\\
               $H_1:\ x_t \sim I(1)$
        \end{itemize}
 \end{itemize}
 DF:
 \begin{align*}
  \Delta x_t=f(t)+\alpha x_{t-1}+u_t
 \end{align*}
 \begin{itemize}
  \item DF assumes $u_t\sim iid$
 \end{itemize}
 Corrections to DF:
 \begin{itemize}
  \item Philips-Perron (PP) test
        \begin{itemize}
         \item Non-parametric correction.
         \item More general but difficult to estimate the $s^2$.
               \begin{itemize}
                \item Allowing a high number of covariates: test statistics looses power
                \item Allowing a low number of covariates?
               \end{itemize}
        \end{itemize}
  \item ADF test:
        \begin{itemize}
         \item Parametric correction.
         \item More efficient for finite samples.
        \end{itemize}
  \item Ng-Perron M test statistics (2001)\\
        3 legs:
        \  \begin{itemize}
         \item[1.] Parametric longrun estimate of $S^2_{AR}$.
         \item[2.] $\bar{\alpha}$ estimated numerically (Elliot, Rothenburg \& Stock, 1996) and Power Envelope Function (PEF) using GLS.
         \item[3.] Number of lags $k$ to include: decided using the Midified Information Criteria (MIC) strategy:
               \begin{itemize}
                \item MAIC
                \item MSIC
               \end{itemize}
        \end{itemize}
 \end{itemize}
 \textbf{ADF-test}\\
 Capturing any information that might be hidden in the $u_t$ that creates and effect in the $\lambda$-term, $\lambda = \sigma^2-\sigma^2_{\varepsilon} / 2$
 \begin{align*}
  \Delta x_t=f(t)+\alpha x_{t-1}+ \sum\limits^{p-1}_{j-1} \gamma_j\Delta x_{t-j} + u_t \\
  \Rightarrow \rightarrow \varepsilon_t\sim iid
 \end{align*}
 \textbf{Philips-Perron (PP) test}\\
 \begin{itemize}
  \item Nice correction to the DF test.
  \item In practice, it's difficult to calculate the short-run variance $s_u^2$ and the long-run  variance $s^2$.
 \end{itemize}
 Complex hypothesis (less feasible):
 \begin{align*}
  \left\{ \begin{array}{cccc}
   H_0: & \alpha=1   & = & x_t\sim I(1) \\
   H_1: & |\alpha=1| & = & x_t\sim I(0)
  \end{array} \right.
 \end{align*}
 Simple alternatives:
 \begin{align*}
  \left\{ \begin{array}{cc}
   H_0: & \alpha=1                                                   \\
   H_1: & \alpha=\bar{\alpha},\ \ \ \bar{\alpha}=1+\frac{\bar{c}}{T}
  \end{array} \right.
 \end{align*}
 i.e. 'local alternatives' in the neighborhood of $H_0$.\par
 Ng-Perron principle:
 \begin{align}
  H_1:\ L(\bar{\alpha})-L(1)
 \end{align}
 Using the likelihood (LM test).
 \\ \\
 \textbf{Def. 'Power Envelope Function' (PEF):}\\
 Chooses $\bar{c}$ such that the $PEF=0.5$\\
 $\Rightarrow$ use \textbf{GLS}.
\end{multicols}


\section{Cointegration Analysis}
\begin{multicols}{2}\noindent
 \textbf{Spurious regression}
 \begin{itemize}
  \item[$\hat{\mu},\hat{\gamma}$] inconsistent (random variables)
  \item[$t_{\hat{\mu}},F^{*}$] inconsistent (diverge towards $\pm\infty$)
  \item[$\bar{R}^2$] inconsistent (random variable)
 \end{itemize}
 \textbf{Best (incomplete) cure}\par
 Model the relationship using differences
 \begin{itemize}
  \item[$\hat{\beta}_{OLS}$] consistent.
  \item[$t_{\hat{\mu}}$] $\sim N(0,1)$
  \item[$F^{*}$] $\sim$ F-snedecor
  \item[$\bar{R}^2$] ok.
 \end{itemize}

\subsection{Estimation of cointegration relationships}
$r=0\Rightarrow$ Spurious regression\\
$0<r<m\Rightarrow$ Cointegration relationships
$r=m\Rightarrow y_t\sim I(0)$
\\ \\
\textbf{Engle \& Granger} estimates a single equation $\Rightarrow r=0, r=1, (0<r<m)?$\\
\begin{itemize}
  \item[\nth{1}] Estimate the LR relationship and obtain the residuals.
  \item[\nth{2}] Cointegration test. ADF $\rightarrow$ critical values for the standard DF test (79; 81).
  \begin{itemize}
    \item $H_0: z_t\sum I(1)\equiv$ No cointegration.
    \item $H_0: z_t\sum I(0)\equiv$ Cointegration.
  \end{itemize}
  \item We only estimate one equation
  \begin{itemize}
    \item[$\rightarrow$] We loose information (regressors might not be exogenous)
    \item[$\rightarrow$] Inefficient estimation, can't do inference.
  \end{itemize}
\end{itemize}
\textbf{Johansen's procedure} allows you to consider all possible equations together.
\begin{itemize}
  \item[$\rightarrow$] All variables in system are assumed to be \textbf{\textit{endogenous}}.
\end{itemize}
\textbf{Consistent + efficient estimation of $\hat{\alpha}$ in single equation framework}
\begin{itemize}
  \item Dynamic Ordinary Least Squares (DOLS) of Saikkonen (1991) and Stock and Watson (1993).
  \begin{itemize}
    \item Works best in a finite sample (in the limit they're the same).
    \item Including $k_2$ lacks and $k_1$ leads.
    \item[$\rightarrow$] A very simply way to include endogeneity in the sample!
    \item Using AIC and BIC to choose the model (might be available in R$\rightarrow$URCA)
  \end{itemize}
  \item Fully Modified (FM) OLS of Philips and Hansen (1990).
  \item Canonical Cointegration Regression (CCR) of Park (1992).
\end{itemize}
$\Rightarrow$ allows to do inference.


\end{multicols}


\section{VAR modelling}
\begin{multicols}{2}\noindent
Introduced by Sims (1980) as an alternative to structural models.
\begin{itemize}
  \item Estimates all possible relations in the system (complete endogeneity).
  \item Allows to model multivariate time series with a minimum of economic theory.
\end{itemize}
Uses
\begin{itemize}
  \item[$\rightarrow$] Delivered very good forecasts without being bound by specification of structure.
  \item[$\rightarrow$] Allows to do 'causality' analysis.
  \item[$\rightarrow$] Impulse response analysis. How would the variables react to a chock to the system?
\end{itemize}
\textbf{VAR(p) model} is a VAR version of the AR(p) model\par
$y_t$ is a vector of variables that depends upon the lag of the vector.\par
There are no ACF or PACF for the VAR(p) model.
\begin{itemize}
  \item[$\rightarrow$] Use an IC.
  \item Use hypothesis testing
  \begin{align*}
   \left\{ \begin{array}{cc}
    H_0: & p=p_0 \\
    H_1: & p=p_1
   \end{array} \right.
  \end{align*}
\end{itemize}



\end{multicols}

%\includegraphics[width = 1.0\textwidth]{CO2.PNG}
