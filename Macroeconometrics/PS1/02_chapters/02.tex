\subsection{Data Generating Process}
A stochastic progress $x_t$ is given by the DGP
\begin{equation}
  \begin{split}
    x_t &= f(t) + \phi x_{t-1}+\varepsilon_t,\\
    \text{with }f(t) &= \mu \neq 0,\ \ \ \ \ |\phi|<1\ \text{and}\ \varepsilon_t\sim iid(0,\sigma^2_\varepsilon)
    \label{eq:xt}
  \end{split}
\end{equation}
$x_t$ is the realization of an ergodic stochastic process if it is stationary and for any point in time $t$ satisfies
\begin{equation}
  \displaystyle{\lim_{T\to\infty}}\left(T^{-1}\sum_{s=1}^TCov(x_t,x_{t+s})\right)=0
\end{equation}
That is, on average the memory of the process is bounded. Thus, covariances goes towards $0$ for two different point in time so that two parts of the same stochastic process can be independent from each other.
\\ \\
By iterative substitution we see that equation \ref{eq:xt} can be rewritten as
\begin{equation}
  \begin{split}
    x_t &= \mu + \phi x_{t-1} +\varepsilon_t \\
        &= \mu + \phi (\mu + \phi x_{t-2}+\varepsilon_{t-1}) +\varepsilon_t\\
        &= \mu (1+\phi) + \phi^2 x_{t-2}+ \phi\varepsilon_{t-1} +\varepsilon_t\\
        &= \mu (1+\phi) + \phi^2 (\mu + \phi x_{t-3}+\varepsilon_{t-2})+ \phi\varepsilon_{t-1}+\varepsilon_t\\
        &= \mu (1+\phi+\phi^2) + \phi^3(x_{t-3})+ \phi^2\varepsilon_{t-2} + \phi\varepsilon_{t-1} +\varepsilon_t\\
        &\vdots \\
        &= \phi^T x_{t-T} + \sum_{s=1}^T \mu\phi^{s-1} + \phi^{s-1}\varepsilon_{t-s+1},\\
        &\ \text{where $T$ is the number of periods $s$ prior to $t$.}
        \label{eq:iterative}
  \end{split}
\end{equation}
Equation \ref{eq:iterative} clearly shows that the memory of $x_t$ is bounded as $|\phi|<1\Rightarrow \phi^T\xrightarrow[T\rightarrow\infty]{}0$ such that the memory of the initial value $x_{t-T}$ and chock $\varepsilon_{t-T+1}$ decreases to zero the further away in time it is from period $t$.

However, the constant $\mu\neq0$ also represents a deterministic trend in the process $x_t$ such that the stochastic process will trend upwards or downwards over time. Thus, $x_t$ in equation \ref{eq:xt} does not describe an ergodic stochastic process.

Applying the difference operator with the order of integration $I(1)$ however would be an ergodic stochastic process as the deterministic parts would cancel each other out.
\begin{equation}
  \begin{split}
    \Delta x_t  &= x_{t} - x_{t-1}\\
                &= [\mu + \phi x_{t-1} +\varepsilon_t] - [\mu + \phi x_{t-2} +\varepsilon_{t-1}]\\
                &= [\mu + \phi (\mu + \phi x_{t-2}+\varepsilon_{t-1}) +\varepsilon_t] - [\mu + \phi x_{t-2} +\varepsilon_{t-1}]\\
                &= [\mu + \phi (\mu + \phi x_{t-2}+\varepsilon_{t-1}) +\varepsilon_t] - [\mu + \phi x_{t-2} +\varepsilon_{t-1}]
        \label{eq:difference}
  \end{split}
\end{equation}
