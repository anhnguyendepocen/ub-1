\epigraph{"..."}
{\textit{.}}

\begin{multicols}{2}
 \section{Introduction}
 Example macro panel data:\par
 Maddison, 2007, and IFS
 \begin{itemize}
  \item Formatting data
  \item Describtive statistics
  \item Graphs
  \item Maps
 \end{itemize}
 % 2
 \section{Fixed Effects estimators} % 2
 \subsection{Simultaneous equations models with exogenous explanatory variables}
 Four different models:
 \begin{itemize}
  \item[M1:] $\alpha\neq,\ \beta\neq.$ Estimation individual by individual (GLS).
  \item[M2:] $\alpha=,\ \beta=.$ Equal constant terms and slopes (presence of homogeneity).
  \item[M3:] $\alpha\neq,\ \beta=.$ Equal slopes, different constant terms.
  \item[M4:] $\alpha=,\ \beta\neq.$ Equal constant terms, different slopes.
 \end{itemize}
 Choosing between them using\par
 \textbf{Test for Homogeneity:}
 \begin{itemize}
  \item[a.] Estimate the extended/unrestricted model.
  \item[b.] Estimate the restricted model.
  \item[c.] $H_0:$ Homogeneity (the unrestricted is not better than the restricted).\\
        Reject $H_0$ if the F-value is higher than the critical value of the F-distribution.
        \begin{align*}
         F = \frac{(SSR_R-SSR_{UR})/r}{SSR_{UR}/df}
        \end{align*}
 \end{itemize}
 In ML we maximize the probability.\\
 In OLS we don't care about the variance.

 \subsection{The fixed effects model}
 $\alpha_i$ is a parameter capturing the individual effect (time-invariant!)
 \begin{align*}
  y_{it}        & =i\alpha_i+X_{it}\beta+\varepsilon_{it}                                            \\
  \Rightarrow y & = [\begin{array}{cc}i & X\end{array}] \left[\begin{array}{c}\alpha \\ \beta\end{array}\right] + \varepsilon
 \end{align*}
 Leasy Squares Dummy Variables (LSDV):
 \begin{itemize}
  \item FE model: All variables are within-transformed e.g the deviations from the mean.
 \end{itemize}
 Test the homogeneity analysis:
 \begin{align*}
  H_0:\ \alpha_1=\alpha_2=\cdots=\alpha_M
 \end{align*}


 \subsection{Within and between estimators}
 The overall variance: Weighted variation between the within-variance and between-variance.


 \subsection{Effects of group and time}


\end{multicols}

% 3
\section{Random effects estimator} % 3
\begin{multicols}{2}
 \subsection{The random effects model}


 \subsection{The generalized least squares estimation}


 \subsection{Feasible Generalized Lest squares (unkown )}



\end{multicols}


4
\section{Fixed effects vs. random estimator} % 4
\begin{itemize}
 \item \textbf{FE allows for correlation} with $y_{it}$ - and has explanatory power.
 \item \textbf{RE doesn't allow for correlation} with $y_{it}$ - the effect is random. Nor does is allow for correlation with $\varepsilon_{it}$.
\end{itemize}
Check if there are individual effects;
\begin{itemize}
 \item $H_0:\alpha=0?$ poolability test / homogeneity test.
 \item $H_0:\sigma_\xi^2=0?$ Breush-Pagan test.
\end{itemize}
\begin{table}[h]
 \centering
 \footnotesize
 \begin{tabular}{l|cc}
  %\toprule
  {}   & No endogeneity           & Endogeneity         \\
  \midrule
  R.E. & Consist. \& Efficient    & Inconsistent        \\
  F.E. & Consist. but inefficient & Consist. but ineff.
 \end{tabular}
 \caption{Endogeneity problems}
 \label{tab:endogeneity}
\end{table}


\begin{multicols}{2}



 \subsection{The Breush Pagan test}


 \subsection{The Hausman test}
 \begin{itemize}
  \item FE - $\beta^{FE}$
  \item FE - $\beta^{RE}$
  \item $()\beta^{FE}-\beta^{RE})$?
 \end{itemize}
 Problems:
 \begin{itemize}
  \item Hausman assumptions
  \item We cannot use "fixed" variables (no time variation).
 \end{itemize}
 \textbf{The Mundlak estimation:}
 \begin{itemize}
  \item[$\rightarrow$] Allows joint estimation F.E. / BE
  \item Diff(FE-BE) $\rightarrow$ estimate 'pseudo' Hausman test.
  \item[$\rightarrow$] Allows for including 'fixed' variables (with no time-variation)
 \end{itemize}
 RE:
 \begin{align*}
  y_{it}+x_{it}\beta+\psi_i+\varepsilon_{it}
 \end{align*}
 Mundlak regression:\par
 Including both RE and BE.
 \begin{align*}
  y_{it}+(x_{it}-\bar{x_i})\beta^w+\bar{x_i}\beta^b +\psi_i+\varepsilon_{it}
 \end{align*}
 Is just an instrumental regression!\par
 Standard errors are unreliable and $R^2$ is lower.
 \begin{itemize}
  \item FE: We capture anything permanent for any individual in $\alpha_i$
        \begin{itemize}
         \item[$\rightarrow$] Get rid of anything permanent.
         \item[$\rightarrow$] Good for controlling.
         \item We don't know what the individual effects mean though!
        \end{itemize}
  \item Mundlak: We 'only' capture anything permanent considered in $\bar{x_i}$
 \end{itemize}
 Increasing the data set (information)
 \begin{itemize}
  \item Hausman test: Is likely to find endogeneity though the difference in coefficients is very small.
 \end{itemize}

 \subsection{Long run vs. short run effects} % 5
 Baltagi \& Griffin (1984) "Short and Long Run Effects in Pooled Models".
 \begin{itemize}
  \item Betweeen est. $\rightarrow$ LR effect.
  \item Within est. $\rightarrow$ SR effect.
  \item OLS and RE $\rightarrow$ average of SR \& LR effect.
 \end{itemize}
 Inequality and growth
 \begin{itemize}
  \item Positive in the SR
  \item Negative in the LR
 \end{itemize}


\end{multicols}


% 5
\section{Heteroskedasticity and autocorrelation in panel data} % 5
\begin{multicols}{2}
 \begin{itemize}
  \item Hausman test
  \item Non-spherical disturbances.
        \begin{itemize}
         \item Heteroscedasticity
         \item Correlation
         \item Autocorrelation
        \end{itemize}
 \end{itemize}
 \subsection{Heteroskedasticity in FE model}

 \begin{align}
  y_{it}= \alpha_i + x_{it}\beta+\varepsilon_{it}
  \label{eq:FE}
 \end{align}
 Heteroscedasticity over time: $\sigma^2_\varepsilon$\\



 \subsection{Heteroskedasticity in RE model}
 \begin{align}
  y_{it}=x_{it}\beta+\xi_i+\varepsilon_{it}
  \label{eq:RE}
 \end{align}
 Heteroscedasticity over individuals: $\sigma^2_\xi$
 Heteroscedasticity over time: $\sigma^2_\varepsilon$\\



 \subsection{Autocorrelation in the FE model}
 In eq. \ref{eq:FE} it can be that
 \begin{align*}
  \varepsilon_{it}=V_{it}+\theta V_{it-1} \\
  \rightarrow \textrm{AR(1)} \rightarrow \varepsilon_{it}=\rho_1\varepsilon_{it-1}+v_{it}
 \end{align*}
 Assumption: Everyone is homogenous in their autocorrelation parameter.\par
 Consistency can come from either a high number of N or T.


 \subsection{Autocorrelation in the RE model}
 We need to start with the FE estimation
 \begin{itemize}
  \item[$\rightarrow$] To get a consistent estimate of $\rho$.
  \item Proceed to estimate either a FE or RE model.
 \end{itemize}


\end{multicols}



% 6
\section{Incomplete/unbalanced panels} % 6
\begin{multicols}{2}
 Incomplete panels are similar to heteroscedasticity issues.
 \begin{itemize}
  \item Some issues can be fixed with weighting observations.
  \item For some procedures we will need complete panels though.
 \end{itemize}

\end{multicols}


% 7
\section{Dynamic panels} % 7
\begin{multicols}{2}
 Became the standard in the 90s for economic growth. Has some issues, so they're not the state of the art anymore. (Can be a column in the regression table).

 Autocorrelation in $y_{it}$.
 \begin{align*}
  y_{it} = \alpha y_{it-1}+\alpha_{it}\beta+\cdots
 \end{align*}

 \subsection{GMM estimation}
 (some have started using ML, but requires the use of strongly nonlinear algorithms)
 \\ \\
 Allows us to impose restrictions\par
 e.g. for $y=x\beta+\varepsilon$:
 \begin{align*}
  cov(x,\varepsilon) & =0 \\
  cov(y,\varepsilon) & =0
 \end{align*}
 This is possible as $u$ and thus $\beta$ is restricted to follow a normal distribution.

 We can use past realizations of y as instruments for $y_{it-1}$
 \\ \\
 \textbf{Limitations to GMM:}
 \begin{itemize}
  \item Need at least 150-200 observations (per time period) $\rightarrow$ or
 \end{itemize}

 \subsection{Validation of instruments}
 Three ways of \textbf{validation} of the instruments
 \begin{itemize}
  \item Hausman Test
  \item Incremental Sargan: Can be problematic (and booring)
  \item Looking at the autocorrelation of the modified residuals - in STATA valid if
        \begin{align*}
         m_1 & : corr(\varepsilon_{it}',\varepsilon_{it-1}')\Rightarrow \text{p-val}<0.05   \\
         m_2 & : corr(\varepsilon_{it-1}',\varepsilon_{it-2}')\Rightarrow \text{p-val}>0.10 \\
             & \text{as }\varepsilon_{it}'=\varepsilon_{it}'-\varepsilon_{it-1}'
        \end{align*}
 \end{itemize}

 \subsection{GMM vs. within estimator}
 \textbf{GMM:} The closer our data is to a normal distribution $\rightarrow$ the faster it converges into normality.
 \\ \\
 The F.E. within estimator:
 \begin{itemize}
  \item Small T $\rightarrow\Rightarrow$ F.E. biased
  \item $T\rightarrow\infty\Rightarrow$ F.E. is ok.
 \end{itemize}

 inference of Z, t is fine, if close to normal and/or observations are high, $N\rightarrow\infty$.

 \subsection{Testing overidentifying restrictions}
Sargan (1958): The error in his test is "proportional to the number of instrumental variables" (counter-intuitive) as the partial $R^2\rightarrow 0.99$
\\ \\
“Forward orthogonal transformation” (Arellano and Bover, 1995) introduce "orthogonal deviations"
 \begin{itemize}
   \item Subtracts the average of all future available observations of a variable.
   \item Instead of subtracting the previous observation from the contemporaneous one
 \end{itemize}
 The basis for the\\
 \\
\textbf{The Blundell Bond (1998) estimator:}\\
"The system GMM estimator". Uses $y_{t-k}$ as instrument for growth $\Delta y$ AND $\Delta y_{t-k}$ as an instrument for levels $y_t$.

\end{multicols}


\section{Endogeneity}

\subsection{The Hausman-Taylor model}
\begin{equation*}
  x_{it} = \underbrace{x_it-\bar{x}_i}_\text{within transformation, time series} - \underbrace{\bar{x}_i}_\text{between transformation, cross-section}
\end{equation*}






%\includegraphics[width = 1.0\textwidth]{CO2.PNG}
