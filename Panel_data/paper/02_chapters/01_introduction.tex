\label{sec:intro}
% • Motivation	of	the	study:	why	you	focus	on	this	particular	issue
% • Hypothesis	and	objective(s)
% • Description	of	the	background	(theoretical	and	empirical)	that	lead
% you	to	propose	the	hypothesis
% • Approach	and	summary	of	results:	what	is	your	strategy	to	check	the
% hypothesis	and	the	main	result
% • Structure	of	the	paper
Innovation driven growth is a topic of high interest both academically and when firms are to decide on their innovation strategies. Using Spanish firm level data from 2005-2016 the objective of this paper is to analyze how different R\&D strategies can contribute differently to increase the sales of new products in the short- or medium-term. Namely the difference in ramp-up time and persistence of internal and external R\&D expenses is analyzed using the different-length leads of sales of new products and Between estimation as well as Fixed Effects and Random Effects estimation\footnote{The STATA do-files can be accessed from \href{https://github.com/thornoe/ub/tree/master/Panel_data/paper/stata_code}{github.com/thornoe/ub/tree/master/Panel\_data/paper/stata\_code}}. Internal R\&D activity is found to have the larger persistent effect while external R\&D spending can be quite efficient when looking for an immediate impact.

A few main findings and the in the literature is summed up as an empirical and theoretical background in section \ref{sec:background} before describing the data, variables and connections in section \ref{sec:data}. The results are presented in section \ref{sec:results} before concluding in \ref{sec:conclusion}.
