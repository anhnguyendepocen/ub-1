\begin{abstract}\noindent
In this paper Spanish manufacturing firms are analyzed to access the timing and impact size of different Research \& Development processes. Lagged and persistent effects are accessed using leads of the share of new products in total turnover of the firm as the dependent variable. The Between, Fixed Effects, and Random Effects estimators are applied and compared to provide robustness. Internal R\&D spending is found to have the larger and more persistent economic impact. However, for firms without internal R\&D expenses investment in external R\&D processes can be a quick fix that helps introduce new products into the market even within the same year, though without building up a know how.
% \\
% \\ \noindent
% \textbf{Keywords}  {\textbullet}  {\textbullet}  {\textbullet}  {\textbullet}
% \\ \\
% \textbf{Keystrokes:} 71.886 \textbf{Standard pages:} 30. \textbf{Contributions:} Thor Donsby Noe: 3.1, 3.2, 4.3, 4.4, 5.1, 5.2, 6.3, 7.1
\end{abstract}
