\label{sec:background}
\subsection{Theory of firm-level innovations}
\label{subsec:theory}
As point of departure I use the same theoretical framework which \citet{sternberg2001firm} use to analyze the probability of innovations in 10 different European regions. As a region level proxy for regional knowledge-spillovers I let the average share of R\&D spending relative to total sales in other firms within the region affect firm-level innovations.

As analyzed by \citet{harrison2014does}, though primarily positive, the effect of R\&D spending and introduction of new products in other firms can be expected to capture effects in two directions: while there can be positive spill-overs through development, interaction and exchange of human capital and technology there is also the risk that sales in other firms might take over market shares. As an elaboration one could expect that
% • Theoretical	arguments	in	the	extant	literature
% only	the	one	more	closely	related	to	your	study

\subsection{Results in other studies}
\label{subsec:other_studies}
\citet{barrios2001explaining} measure the effects of R\&D spendings and regional spillovers on firms' export from Spain. \citet{harrison2014does} estimate the effects of innovations on employment growth in Spanish firms using the PITEC database. \citet{vogel2015two} find effects of R\&D spendings and evidence of local technology spillovers through her estimation of a long run model for convergence in Total Factor Productivity (TFP) between all European regions. Though they find that both region-level variables and firm-level variables are significant, \citet{sternberg2001firm} found regional characteristics and regional between-firm spillovers to be of less importance than the characteristics of the individual firm itself.
% Previous	empirical	evidence
% only	the	one	more	closely	related	to	your	study!
