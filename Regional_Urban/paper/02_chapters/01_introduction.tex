\label{sec:intro}
% • Motivation	of	the	study:	why	you	focus	on	this	particular	issue
% • Hypothesis	and	objective(s)
% • Description	of	the	background	(theoretical	and	empirical)	that	lead
% you	to	propose	the	hypothesis
% • Approach	and	summary	of	results:	what	is	your	strategy	to	check	the
% hypothesis	and	the	main	result
% • Structure	of	the	paper
In a global competition a firm's ability to innovate and come up with new and successful products is crucial for the development of the firm as well as regional growth and jobs.

Innovations is not just the result of isolated research and development (R\&D) processes within firms but can also be expected to be driven up by knowledge spillovers from a high share of R\&D spending in other firms within the same region by the means of formal and informal interactions between firms and employees of different firms. Though I do not find significant spillovers from regional R\&D expenses a few of the regional dummies correct some of the heteroscedasticity in the estimation. The internal R\&D activity seem to be the more important, especially the continuity of in-house basis research.

I proceed by outlining the theoretical background in section \ref{sec:background}, the empirical strategy and the data in section \ref{sec:empirics} and \ref{sec:data} respectively, before presenting the results in \ref{sec:conclusion} and concluding in \ref{sec:conclusion}.
