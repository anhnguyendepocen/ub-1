\label{sec:data}
% Dataset	and	variables
% • Main	characteristics	of	the	dataset:	source,	type	of	data,	…
% • Description	of	variables	used	for	the	analysis	and	correspondence
% with	the	(ideal)	magnitudes	in	the	empirical	specification
% • Descriptive	statistics	of	the	main	variables	in	the	analysis
Using the PITEC database\footnote{Description of the methodology, the full questionnaires etc. in PITEC is available by the Spanish Foundation of Science \& Technology at \href{https://icono.fecyt.es/pitec}{icono.fecyt.es/pitec}} the Spanish region of the firm is identified using the firm's location based on information on the region in which they have at least 75\% of their R\&D department.\footnote{This novel identification strategy for regions is developed by Enrique López-Bazo.} This implies that the modified dataset will be limited to single-unit firms that have some internal R\&D personnel employed for at least one of the years.
\\
\\
Dependent variables
\begin{itemize}
  \item Sales of new products (share of total sales constituted by products, invented within the past two years).
  \begin{itemize}
    \item Due to including products invented within the year as well as in the past two years, this variable is included as a lead one year ahead.
  \end{itemize}
\end{itemize}
Main explanatory variables
\begin{itemize}
  \item Regional level as well as firm level variables on R\&D’s share of total expenditure. For the regional level the average share is calculated for each firm after deducting the R\&D expenses and total revenue for the firm.
  \item Whether there is continuity in R\&D activity or not. Disaggregated into basic and applied research as well as investments in technological development.
\end{itemize}
Controls
\begin{itemize}
  \item Regional dummies.
  \item Year dummies, not shown in the regression table.:
  \item	Statistical Classification of Economic Activities (CNAE93), not shown in the regression table.
  \begin{itemize}
    \item The manufacturing sector and service sector respectively is split into six industries for each.
    \item Other sectors are dropped.
\end{itemize}
\end{itemize}
