\epigraph{"As economists we only see a part of the picture"}
{\textit{Monica Serrano Gutierrez}}

\section{Warming up}
\noindent
\begin{multicols}{2}
\subsection{Constanza et al (2015) Time to leave GDP behind}
by Costanza, R., I.Kubiszewski, E. Giovanini, H. Lovins, J. McGlade, K.E. Pickett, K.V. Ragnarsdóttir, D. Roberts, R. de Vogli \& R. Wilkinson (2014), Nature (\href{http://www.nature.com/news/development-time-to-leave-gdp-behind-1.14499}{link})
\epigraph{GDP measures "everything except that which makes life worthwhile"}
{\textit{Robert F. Kennedy}}
\noindent
GDP is a good measure for tne flow of everything that has a market price - mot as an indicator of well-being or environment.
\\
\textbf{Alternative measures} should take into account
\begin{itemize}
  \item Happiness
  \item Prosperity
  \item Environment
  \item Development
\end{itemize}
\subsection{Rodrik, D. (2015) Economics Rules: The Rights and Wrongs of the Dismal Science }
\noindent
An economist should have as many different models as possible in her toolbox\\
$\rightarrow$ choose the better model(s) for the specific research question.
\\
Our models are partial, thus, our conclusions are partial.

\subsection{The four laws of thermodynamis}
\begin{itemize}
  \item[\nth{1}] Law of thermodynamis: Energy can neither be created nor destroyed, but can change formms and flow from one place to another.
  \item[\nth{2}] Law of thermodynamis: The irreversibility of natural processes, and, in many cases, the tendency of natural processes to lead towards spatial homogeneity of matter and energy.
\end{itemize}
\noindent
Important works on environmental economics
\begin{itemize}
  \item Pigout (1920): Taxing externalities.
  \item Coase (NPE 1991): Contracting between parties.
  \item Elinor Ostrom (NPE 2012): Some communities use other mechanisms than the market for allocations etc. - and it's better than the market!
  \item Richard H. Thaler (NPE 2017): Behavioral economics (interests of firms).
  \item William Nordhaus (NPE 2018): For integrating climate change into long-run macroeconomic analysis.
\end{itemize}
\subsection{Environmental Economics vs. Ecological Economics}
\epigraph{"We cannot solve our problems with the same thinking we used when we created them"}{\textit{Albert Einstein}}
\noindent
Ecological Economics
\begin{itemize}
  \item Sustainability of the world as a whole.
  \item Looking at the world as a whole, i.e. no such thing as externalities.
\end{itemize}
Environmental Economics
\begin{itemize}
  \item Sustainability: Of the economy.
  \item Negative externalities: To the economy (the core).
  \begin{itemize}
    \item Uncompensated (adverse) impact of one person's action on the wellbeing of a bystander.
    \item Causes markets to be inefficient, and thus to maximize total surplus, e.g. pollution.
    \item Coase theorem: if private parties can bargain without cost over the allocation of resources, they can solve the problem of externalities on their own.
    \item Government action: Regulations (permits) or taxations (market correcting solution).
  \end{itemize}
\end{itemize}
\noindent
\textbf{The Climate:}\\
Average weather conditions that can be observed locally regionally or globally. Changes with or without human impact.
\\ \\
\textbf{Global warming:}
\begin{itemize}
  \item This is what is important!
  \item Designates the increase of average temperature
  \item \textbf{Global public good:}\\
  Standard solutions to tragedy of the commons:
  \begin{itemize}
    \item Price market-based policy: Carbon tax: Arthur Pigou (1920) \textit{The Economics of Welfare}
    \item Quantity market-based policy: Cap-and-trade system: Ronald Coase (1920) \textit{The problem of social cost}
    \item Alternative methods: Polycentric approach (consensus): Elinor Ostrom (2012) \textit{GLobal Environmental Commons} (NP, 2009).
  \end{itemize}
  Options to manage the "global common"
  \begin{itemize}
    \item Free rider problem: Westphalian nature of the current system of nations
    \item Problem of responsibiilty
  \end{itemize}
\end{itemize}
\textbf{History of international climate negotiations}\\
1987: Montreal: Agreement about the Freon gas - the only succesfull negotiation.
\end{multicols}

 % 2
\section{The Economy as an Open System} % 2
\begin{multicols}{2}
sdk


\end{multicols}

% 3
\section{Price input-output model} % 3
\begin{multicols}{2}
Great because flows can be in all kinds of measures - we don't need to translate everything into Euroes.\\ \\ \\ \\ \\


\end{multicols}
% 4
\section{International Databases for the economy and the environment} % 4
\begin{multicols}{2}
sdk


\end{multicols}


% 4
\section{...} % 4
\begin{multicols}{2}


\end{multicols}


%\includegraphics[width = 1.0\textwidth]{CO2.PNG}
