\label{sec:conclusion}
The elasticity of taxable income (EIT) with respect to the marginal tax rate (MTR) not only accounts for labour supply but also other behavioral responses such as tax avoidance, tax evasion, collective agreements and career choices.

Though differences-in-differences estimation using OLS can be a simple way to analyze effects in the proximity of a substantial tax reform, it can be difficult to completely exclude effects from non-tax-related changes to inequality and mean reversion. Availability of controls as well as 2SLS panel regression over a period with a variety of tax system changes can reduce these biases.

While most studies for the U.S. estimate a significant ETI, studies for Denmark tend to find modest or zero effects of changes in the MTR for the wage earner when being able to control for intertemporal income-shifting or self-employed. On one hand the discrepancy between the ETI in the U.S. and Denmark can partly be explained by higher frictions and less options for tax avoidance, on the other hand estimates have also decreased with richer better data availability allowing for more controls. This suggests that future studies in the U.S. might also find estimates closer to zero come richer data availability, but also that there might be a revenue and efficiency loss from gaps in the U.S. tax law.
